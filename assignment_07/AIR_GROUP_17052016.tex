\documentclass[11pt]{article}

\usepackage[utf8]{inputenc}
\usepackage{graphicx}
\usepackage{hyperref}
\title{\textbf{Artificial Intelligence for Robotics\\ - Homework 7 -}}
\author{Kiran Vasudev, Patrick Nagel}
\date{Due date: 24.05.2016}
\begin{document}

\maketitle

\newpage
\tableofcontents

\newpage
\section{Answer questions}
\subsection{What are local search algorithms?}

\subsection{What are the advantages of local search?}
\begin{itemize}
\item {These search algorithms use very little memory.}
\item {These algorithms can find a good solution in a very large or infinite state space}
\end{itemize}

\subsection{When do we try to find the global minimum?}

\subsection{When do we try to find the global maximum?}
We find the global maximum when the elevation(value of the heuristic function) corresponds to an objective function. An objective function is one that assigns a value to a solution or a partial solution.

\subsection{What is the characteristic of a complete local search algorithm?}

\subsection{What is the characteristic of an optimal algorithm?}
The characteristic of an optimal algorithm is that it finds the best possible outcome(\textbf{global maximum/minimum}) from all possible outcomes in a state space.

\subsection{What is a landscape?}
\subsection{What is Hill Climbing?}
Hill Climbing is an algorithm that works by moving continuously in the direction of increasing value of traversing from one node in the state space to another(uphill). This algorithm terminates when there is no other neighbour that has a higher value than the current neighbour, ie. when it reaches its peak.

\subsection{What is the problem of Hill Climbing?}
\subsection{What drives the success of Hill Climbing?}
The success of Hill Climbing algorithm depends mainly on the point/node at which the algorithm starts.

\subsection{What is Simulated Annealing?}
\subsection{What is the condition that enables Simulated Annealing to find
the optimal solution?}
The main condition that makes this algorithm find the optimal solution is that Simulated Annealing takes a random move and checks if this random move improves the current situation. If it improves the current situation, then the random move is accepted. In Hill Climbing, the algorithm picks the best value, whereas in Simulated Annealing, the algorithm finds a random value which is better than the current value. 

\newpage
\section{Discussion and comments on the Travelling Salesman Problem}
The Travelling Salesman Problem is a problem where every city must be visited exactly once. Our aim is to find the shortest path to visit every single city. The Travelling Salesman Problem is an NP-Hard problem. \\

In our approach we are looking for closest city to the current city. Tests showed that even after 200 iteration the differences between the total distances are not that big. Each iteration starts with a different random city. This way we can avoid getting stuck in a local minimum.
\end{document}